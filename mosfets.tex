\documentclass[letterpaper,10pt]{report}
\usepackage[utf8x]{inputenc}

\usepackage{amsmath}

\title{MOSFET Theory for Amateur Radio}
\author{Philip G. Lee (NU9J)}

\begin{document}
\maketitle

This document is Copyright Philip G. Lee 2013 under the GFDL 1.3.
Permission is granted to copy, distribute and/or modify this document
under the terms of the GNU Free Documentation License, Version 1.3 
or any later version published by the Free Software Foundation;
with no Invariant Sections, no Front-Cover Texts, and no Back-Cover Texts.
A copy of the license is included in the file \texttt{COPYING.GFDL.tex}.

\tableofcontents

\chapter*{Preface}%============================================================

I have found no book explaining basic MOSFET circuits used in radio frequency
applications. This is my attempt to create one with the knowledge that was so
painstakingly learned and derived.

\chapter{MOSFET Theory}%=======================================================

\section{Notation}

\begin{table}
 \begin{tabular}{l|l}
  Symbol & Meaning \\
  \hline
  $V_{DD}$ & Supply voltage applied to drain\\
  $V_{GS}$ & DC voltage between $G$ and $S$\\
  $v_{GS}$ & AC voltage between $G$ and $S$
 \end{tabular}
 \caption{Symbol meanings.}
 \label{tab:symbols}
\end{table}

Capital letters $V$ and $I$ mean DC voltages and currents respectively.
Subscripts denote reference points for the measurement. Repeated subscripts
denote power sources. Examples are given in Table \ref{tab:symbols}.

\section{Basic Theory}

These approximations are usually valid:
\begin{align}
 V_{th} &\approx 2V
\end{align}

\section{Saturation Mode}

The conditions for operation in saturation mode are:
\begin{align}
 v_{GS} &\geq V_{th} \; \text{and}\\
 v_{DS} &\geq v_{GS} - V_{th}. \label{eq:saturationCond}
\end{align}

In this mode, a simplified model for the transistor's operation is:
\begin{align}
 i_D = K(v_{GS}-V_{th})^2 \label{eq:saturationOperation}
\end{align}

To find the saturation region on the $(v_{DS},i_D)$
plots, we note that Eq. \eqref{eq:saturationCond} gives
$v_{GS} \leq V_{th}+v_{DS}$. Substituing that into Eq. \eqref{eq:saturationOperation},
we have:
\begin{align}
 i_D \leq Kv_{DS}^2. \label{eq:saturationCond2}
\end{align}

\subsection{Transconductance}

Typically, the transistor is biased inside the saturation region with a DC
gate voltage $V_{GS}$, causing a DC current $I_D=K(V_{GS}-V_{th})^2$ to flow
into the gate. When an AC voltage $v_{in}$ is superimposed
($v_{GS}=V_{GS}+v_{in}$), we can rewrite Eq. \eqref{eq:saturationOperation}
with a Taylor expansion around the bias point as:
\begin{align}
 i_D = I_Q + 2\sqrt{KI_Q} v_{in} + K v_{in}^2, \label{eq:qpointOperation}
\end{align}
where $I_Q=K(V_{GS}-V_{th})^2$ is the DC \textbf{quiescent current}.

Since Eq. \eqref{eq:qpointOperation} tells us that a small change in $v_{in}$
results in a linear increase in current (ignoring the squared term), the
transconductance of the device is:
\begin{align}
 g_m = 2\sqrt{KI_Q} = 2K(V_{GS}-V_{th}).
\end{align}

In practice, you can get $g_m$ directly off the spec sheet if it includes a
$(v_{DS},i_D)$ plot with curves of closely-spaced $v_{GS}$. For example,
if it shows that for $v_{GS}=4.5V$, $i_D=0.3A$, and for $v_{GS}=5.0V$,
$i_D=1.0A$, then $g_m \approx (1.0A-0.3A)/(5.0V-4.5V) = 1.4A/V$.

\chapter{Generic RF Amplifier}%================================================

\noindent (Schematic of biased NFET w/ drain RFC to VDD and load resistance)

\begin{align}
 v_{GS} &= V_{GS} + v_{in} \\
 v_{DS} &= V_{DD} - i_LR_L \\
 i_L &= i_D-I_Q
\end{align}

We want this amplifier to be as linear as possible, operating in the saturation
mode, and to develop as much power as possible. One of the limiting factors
is the requirement to stay in the saturation region. Equation
\eqref{eq:saturationCond2} becomes:
\begin{align}
 i_D &\leq Kv_{DS}^2 \implies \nonumber \\
 i_L + I_Q &\leq K(V_{DD}-i_LR_L)^2. \nonumber
\end{align}
To make the analysis simpler and independent of $I_Q$, we make the assumption
that it is nearly 0.
\begin{align}
 &i_L \leq K(V_{DD}-i_LR_L)^2 \; \text{i.e.} \nonumber \\
 &K(V_{DD}-i_LR_L)^2 - i_L \geq 0.
\end{align}

To develop the most power, we want to find the maximum current $i_{L(max)}$ that
satisfies the inequality. This is a really messy quadratic equation to solve
exactly for $i_L$, but you might approximate it by $i_{L(max)}=V_{DD}/R_L$. The
problem with that estimate is that it overestimates the maximum current by
making the squared term to go 0 and violating the constraint by $-V_{DD}/R_L$.
A better estimate is
\begin{align}
 i_{L(max)} \approx \frac{V_{DD}}{\sqrt{2}R_L},
\end{align}
which usually underestimates the maximum current by a bit (somehow
\textbf{NOTE}: need to find out and explain why).

So, the maximum power we can develop by operating in the saturation mode is
\begin{align}
 P \approx i_{L(max)}^2R_L = \frac{V_{DD}^2}{2R_L}.
\end{align}
This gives us an easy way to set $R_L$ for a given power
\begin{align}
 R_L = \frac{V_{DD}^2}{2P}.
\end{align}


\section{Distortion}

Distortion (nonlinearity) is reduced by increasing $I_Q$.

\section{Design Procedure}

\begin{enumerate}
 \item Choose $I_Q$ big enough to meet linearity requirements.
 \item Get $g_m$ from spec sheet.
 \item Calculate $K = g_m^2/(4I_{g_m})$, where $I_{g_m}$ should be given along with
       $g_m$ on spec sheet.
 \item Calculate the maximum power this FET can deliver.
  \begin{align}
   P_{max} = \frac{V_{DD}}{\sqrt{2}}(KV_{DD}^2(\sqrt{2}-1)^2/2-I_Q)
  \end{align}
 \item Choose desired power $P \leq P_{max}$.
 \item Set $R_L=V_{DD}^2/(2P)$.

 \item Calculate required input voltage $v_{in(max)} = g_m^{-1}\sqrt{P/R_L}$.
 \item Add input step-up or step-down network as necessary.
 \item Check that the MOSFET is still in saturation: $i_D \leq K v_{DS}^2$.
  \begin{align}
   I_Q + g_m v_{in(max)} &\leq K (V_{DD}-g_m v_{in(max)}R_L)^2 \nonumber \\
   I_Q + \sqrt{P/R_L} &\leq K(V_{DD}-\sqrt{P R_L})^2 \nonumber \\
   I_Q + \sqrt{2} P/V_{DD} &\leq K(V_{DD}-V_{DD}/\sqrt{2})^2 \nonumber \\
   I_Q + \sqrt{2} P/V_{DD} &\leq KV_{DD}^2(1-1/\sqrt{2})^2 \nonumber \\
  \end{align}
 \item Add matching network to transform $R_L$ to required output impedance
\end{enumerate}

\end{document}          
