\documentclass[letterpaper,10pt]{report}
\usepackage[utf8x]{inputenc}

\usepackage{amsmath}

\title{MOSFET Theory for Amateur Radio}
\author{Philip G. Lee (NU9J)}

\begin{document}
\maketitle

This document is Copyright Philip G. Lee 2013 under the GFDL 1.3.
Permission is granted to copy, distribute and/or modify this document
under the terms of the GNU Free Documentation License, Version 1.3 
or any later version published by the Free Software Foundation;
with no Invariant Sections, no Front-Cover Texts, and no Back-Cover Texts.
A copy of the license is included in the file \texttt{COPYING.GFDL.tex}.

\tableofcontents

\chapter*{Preface}%============================================================

I have found no book explaining basic MOSFET circuits used in radio frequency
applications. This is my attempt to create one with the knowledge that was so
painstakingly learned and derived.

\chapter{MOSFET Theory}%=======================================================

\section{Notation}

\begin{table}
 \begin{tabular}{l|l}
  Symbol & Meaning \\
  \hline
  $V_{DD}$ & Supply voltage applied to drain\\
  $V_{GS}$ & DC voltage between $G$ and $S$\\
  $v_{GS}$ & AC voltage between $G$ and $S$
 \end{tabular}
 \caption{Symbol meanings.}
 \label{tab:symbols}
\end{table}

Capital letters $V$ and $I$ mean DC voltages and currents respectively.
Subscripts denote reference points for the measurement. Repeated subscripts
denote power sources. Examples are given in Table \ref{tab:symbols}.

\section{Basic Theory}

These approximations are usually valid:
\begin{align}
 V_{th} &\approx 2V
\end{align}

\section{Saturation Mode}

The conditions for operation in saturation mode are:
\begin{align}
 v_{GS} &\geq V_{th} \; \text{and}\\
 v_{DS} &\geq v_{GS} - V_{th}. \label{eq:saturationCond}
\end{align}

In this mode, a simplified model for the transistor's operation is:
\begin{align}
 i_D = K(v_{GS}-V_{th})^2 \label{eq:saturationOperation}
\end{align}

To find the saturation region on the $(v_{DS},i_D)$
plots, we note that Eq. \eqref{eq:saturationCond} gives
$v_{GS} \leq V_{th}+v_{DS}$. Substituing that into Eq. \eqref{eq:saturationOperation},
we have:
\begin{align}
 i_D \leq Kv_{DS}^2.
\end{align}

\subsection{Transconductance}

Typically, the transistor is biased inside the saturation region with a DC
gate voltage $V_{GS}$, causing a DC current $I_D=K(V_{GS}-V_{th})^2$ to flow
into the gate. When an AC voltage $v_{in}$ is superimposed
($v_{GS}=V_{GS}+v_{in}$), we can rewrite Eq. \eqref{eq:saturationOperation}
with a Taylor expansion around the bias point as:
\begin{align}
 i_D = I_Q + 2\sqrt{KI_Q} v_{in} + K v_{in}^2, \label{eq:qpointOperation}
\end{align}
where $I_Q=K(V_{GS}-V_{th})^2$ is the DC \textbf{quiescent current}.

Since Eq. \eqref{eq:qpointOperation} tells us that a small change in $v_{in}$
results in a linear increase in current (ignoring the squared term), the
transconductance of the device is:
\begin{align}
 g_m = 2\sqrt{KI_Q} = 2K(V_{GS}-V_{th}).
\end{align}


\chapter{Generic RF Amplifier}%================================================

\section{Distortion}

\section{Design Procedure}

\end{document}          
