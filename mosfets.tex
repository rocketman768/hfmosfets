\documentclass[letterpaper,10pt]{report}
\usepackage[utf8x]{inputenc}

\usepackage{amsmath}

\title{MOSFET Theory for Amateur Radio}
\author{Philip G. Lee (NU9J)}

\begin{document}
\maketitle

This document is Copyright Philip G. Lee 2013 under the GFDL 1.3.
Permission is granted to copy, distribute and/or modify this document
under the terms of the GNU Free Documentation License, Version 1.3 
or any later version published by the Free Software Foundation;
with no Invariant Sections, no Front-Cover Texts, and no Back-Cover Texts.
A copy of the license is included in the file \texttt{COPYING.GFDL.tex}.

\tableofcontents

\chapter*{Preface}%============================================================

I have found no book explaining basic MOSFET circuits used in radio frequency
applications. This is my attempt to create one with the knowledge that was so
painstakingly learned and derived.

\chapter{MOSFET Theory}%=======================================================

\section{Notation}

\section{Basic Theory}

These approximations are usually valid:
\begin{align}
 V_T &\approx 2V
\end{align}

\section{Saturation Mode}

The conditions for operation in saturation mode are:
\begin{align}
 v_{GS} &> V_T \; \text{and}\\
 v_{DS} &> v_{GS} - V_T.
\end{align}

In this mode, a simplified model for the transistor's operation is:
\begin{align}
 i_D = K(v_{GS}-V_T)^2
\end{align}

\chapter{Generic RF Amplifier}%================================================

\end{document}          
